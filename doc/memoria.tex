%!TEX TS-program = xetex
%!TEX encoding = UTF-8 Unicode

%% -----------------------------------------------
%% Fichero de memoria de la práctica 4 de Sistemas
%%    Empotrados y de Tiempo Real, EII - ULPGC
%% -----------------------------------------------

\documentclass[spanish]{article}

\usepackage[utf8]{inputenc}
\usepackage{babel}
\usepackage{hyperref}
\usepackage{listings}
\usepackage{color}
\usepackage[margin=0.8in]{geometry}
\usepackage{float}
\usepackage{dirtytalk}
\usepackage{amsmath}
\usepackage{graphicx}
\usepackage{fontspec}
\graphicspath{ {images/} }

\lstset { %
  language=C++,
    commentstyle=\color{green},
    keywordstyle=\color{blue},
    basicstyle=\footnotesize
}
\inputencoding{utf8}
\hypersetup{
  pdfborder={0 0 0},
  linkcolor=blue
}
\newcommand{\colorhref}[3] {
  \color{#1}{\href{#2}{#3 }}\color{black}
}

%\newfontfamily{\os}{Open Sans}
%\newfontfamily{\osl}[UprightFont={* Light}]{Open Sans}

\setsansfont
  [Ligatures=TeX, % recommended
   UprightFont={* Regular}]
  {Open Sans}

\begin{document}

\begin{titlepage}
  \centering
  \begin{figure}
    \centering
    \def\svgwidth{0.20\columnwidth}
    \input{images/ULPGC.pdf_tex}
  \end{figure}
  {\scshape\Large Universidad de Las Palmas de Gran Canaria \par}
  \vspace{1cm}
  {\scshape\large Sistemas Empotrados y de Tiempo Real\par}
  \vspace{2.5cm}
  {\huge\bfseries Práctica Final\par}
  \vspace{1.5cm}
  {\Large\itshape por Diego Sáinz de Medrano\par}

  \vfill

  % Bottom of the page
  {\large \today\par}
\end{titlepage}


  \newpage
  \pagenumbering{arabic}
  \section{Introducción}
  \section{Tecnologías usadas}
 	\subsection{Arduino-Ide}
  \begin{figure}[H]
    \centering
    u mad tho lol rly ya
  \end{figure}
  \section{Ficheros entregados}
  \begin{itemize}
    \item{}
  \end{itemize}

  \vfill
  \section{Bibliografía}
  \begin{itemize}
    \item{\colorhref{blue}{http://people.sc.fsu.edu/~7Ejburkardt/data/ply/f16.ply}{http://people.sc.fsu.edu}: de donde hemos descargado el modelo PLY del F16.}
    \item{\colorhref{blue}{https://stackoverflow.com/}{Stack Overflow}: para dudas generales sobre C++, OpenGL y Arduino.}
    \item{\colorhref{blue}{http://telepresencial1617.ulpgc.es/cv/ulpgctp17/course/view.php?id=428}{Transparencias de teoría y prácticas de CIU}.}
    \item{\colorhref{blue}{https://github.com/acoffman/solar-system-opengl/tree/master/Bitmaps}{Github : }Texturas .bmp para el sistema solar.}
  \end{itemize}

\end{document}
